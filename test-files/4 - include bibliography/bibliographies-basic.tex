\documentclass[12pt]{article}
\PassOptionsToPackage{hyphens}{url}\usepackage{hyperref}

\begin{document}

% bibliography managers
% - zotero
% - onenote
% - mendeley
% - jabref -> LaTeX specific package (http://www.jabref.org/)

% make random citations
\section{Adding citations}
One of the shortest articles \cite{bib_entry1} is followed by a random citation I made \cite{whatever_name_you_want}. Sometimes most references are from the web which you can equally cite \cite{cool_paper_from_web}. Sometimes you need to cite more than one \cite{bib_entry1,whatever_name_you_want,cool_paper_from_web}

% create bibliographies manually
% go to google scholar
% step 1 - find article you want
% step 2 - click on the quotation mark icon at the bottom of the google entry
% step 3 - highlight and copy whatever bib format is shown in the pop-up
% step 4 - past beneath a \bibitem entry below
% step 5 - give the \bibitem a unique name to cite
% change the bibliography style
%------------------------%
\bibliographystyle{plain}
%------------------------%

\begin{thebibliography}{99} % <- doesn't mean 99 entires, exactly, but the maximum amount of anticipated entries
\bibitem{bib_entry1}
Lander, L. J., and T. R. Parkin.``Counterexample to Euler’s conjecture on sums of like powers'' Bull. Amer. Math. Soc 72.6 (1966): 1079.

\bibitem{whatever_name_you_want}
I just made up my own bibliography format

\bibitem{cool_paper_from_web}
\url{https://tex.stackexchange.com/questions/3033/forcing-linebreaks-in-url}

\end{thebibliography}
\end{document}