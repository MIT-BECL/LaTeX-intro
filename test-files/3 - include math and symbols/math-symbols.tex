\documentclass[12pt]{article}

% essential math based packages 
\usepackage{amsfonts}	% commonly used fonts and symbols in mathematics
\usepackage{amsmath}	% additional set of math tools on top of LaTeX
\usepackage{amssymb}	% common math symbols
\usepackage{amsthm}	% theorem based setup following AMS standards

\usepackage[T1]{fontenc}

\usepackage{multicol}
 
\usepackage{hyperref} 
\hypersetup{
    colorlinks=true,
    linkcolor=blue,
    filecolor=magenta,      
    urlcolor=cyan,
}
 
\begin{document}

% remove indentations for entire document
\setlength\parindent{0pt}

\section{Math enviornments}
% Introduciont to math enviornments
%------------------------------------%
% short cut symbols
\subsection*{Inline elements}
\begin{enumerate}
	\item using \verb!$...$!: $a+b=c$ 
	\item using \verb!\[...\]!: \(a/b=c\) 
	\item using environment \verb!\begin{math}...\end{math}!): 
		\begin{math}
		a - b = c
		\end{math}
\end{enumerate}
Takeaway: Although there are said to be issues, or incompatibilities with using \verb!$...$!, most examples and working code use this shorthand for inline math. If for whatever reason you do find issues, then use \verb!\(...\)!.

\subsection*{Blocked elements}
\begin{enumerate}
	\item using \verb!$$...$$!: $$ \frac{a}{b}=c $$
	\item using \verb!\[...\]!: \[ \int^a_b = c\]
	\item using enviornment \verb!\begin{displaymath}...\end{displaymath}!:
		\begin{displaymath}
		\dfrac{\partial a}{\partial	b} = c
		\end{displaymath}
\end{enumerate}
Takeaway: Much like for inline math, \verb!$$...$$! is commonly used. However, if you do find issues (which rarely does), then use \verb!\[...\]!. 

\subsection*{Alignments \& numberings}
\begin{enumerate}
	\item Numbered equations
	\begin{equation}
		\text{KE} = 1/2mv^2
	\end{equation}
	
	\item No numbered equations ({\*}trick, similar to section numberings)
	\begin{equation*}
		\text{PE} = \int_{\text{ref}}^{x} F \operatorname{d}\!\overrightarrow{x}
	\end{equation*}	
	
	\item Numbered equations (not aligned)
	\begin{gather}
		\exp^{ix} = \cos{x} + i\sin(x) \\
		\exp^{i\pi} + 1 = 0
	\end{gather}
	
	\item Numbered and aligned
	\begin{align}
		\nabla \cdot \vec{D} &= \rho_v \\
		\nabla \cdot \vec{B} &= 0 \\
		\nabla \times \vec{E} &= - \frac{\partial B}{\partial t} \\
		\nabla \times \vec{B} &= \mu_{0}\vec{J} +
		\mu_{0}\epsilon_{0}\frac{\partial E}{\partial t}
	\end{align}
	
	\item Controlling numbering and alignment
	\begin{align}
		&\nabla \cdot \vec{D} = \rho_v \nonumber \\
		&\nabla \cdot \vec{B} = 0 \\
		&\nabla \times \vec{E} = - \frac{\partial B}{\partial t} \\
		&\nabla \times \vec{B} = \mu_{0}\vec{J} +
		\mu_{0}\epsilon_{0}\frac{\partial E}{\partial t} \nonumber
	\end{align}
\end{enumerate}
Takeaway: All environments can use the \* trick to suppress numbering, or \verb!\nonumber! can do this specifically per line. Although not shown in this demonstration, if equations get too long, or multiple equations should be given 1 equation number (such as an \textit{if/else} statement), use the \verb!\begin{split}! or \verb!\begin{multiline}! environments.

%------------------------------------%

\section{Symbols}

\subsection*{Greek symbols}
Note that greek symbols that can be represented by english letters such as \verb!\Alpha! and \verb!\Chi! do not exists, as their symbols $A$ and $X$ are indistinguishable from using letters \verb!$A$! and \verb!$X$!. However, some packages override this behavior, so please check what math packages you import.

$$\alpha, A, \beta, B, \gamma, \Gamma, \delta, \Delta ...\, \mu, \nu $$

\subsection*{Equation symbols}
You have control over all types of symbols relevant to mathematical, and even graphical representation. To get an extensive list please look \href{https://en.wikibooks.org/wiki/LaTeX/Mathematics#List_of_mathematical_symbols}{here}.

\subsection*{Formatting mathematical symbols}
Some equations need more than a simple definition or symbol. Some symbols can be compounded to make more complex statements. For example
$$
\overrightarrow{\sum_{i=\iiint}^{j={\widehat{AAA}}}}
$$
A more comprehensive discussion on this topic, and how to customize the look can be found \href{https://en.wikibooks.org/wiki/LaTeX/Mathematics#Formatting_mathematics_symbols}{here}.

\section{Spacing}
Horizontal spacing is dictated by the document class font size (e.g. 11pt, 12pt, etc.) and is measured by \textit{em} which is roughly proportional to the horizontal width of a capital M. To artificially create 1 em width is to use \verb!\quad!. See: \\

\indent A{\quad}B\\
\indent AMB (... a little bit more than M)\\

Knowing this, there are many commands such as \verb!\,! \verb!\:! that create fractions of \verb!\quad!. The variety of commands for horizontal spacing in normal and math mode can be found \href{https://en.wikibooks.org/wiki/LaTeX/Mathematics#Controlling_horizontal_spacing}{here}. There is also a discussion on which \href{https://tex.stackexchange.com/questions/41476/lengths-and-when-to-use-them}{spacing is appropriate}.


\end{document}